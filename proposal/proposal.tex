\documentclass[conference]{IEEEtran}
\makeatletter
\newcommand{\rmnum}[1]{\romannumeral #1}
\newcommand{\Rmnum}[1]{\expandafter\@slowromancap\romannumeral #1@}
\makeatother

\usepackage{epsfig}
\usepackage{amsopn}
\usepackage{subfigure}
\usepackage{cite}
\ifCLASSINFOpdf

\else
\fi

\usepackage[cmex10]{amsmath}

\interdisplaylinepenalty = 2500

\usepackage{algorithmic}
\usepackage{algorithm}
\hyphenation{op-tical net-works semi-conduc-tor}


\begin{document}
\title{Research Proposal}
\author{\authorblockN{Quan~Wang\authorrefmark{1}, Tingwu~Wang\authorrefmark{1}, Wenxin~Wang\authorrefmark{1}, and Tianpen~Li\authorrefmark{1}}
      \small\authorblockA{\authorrefmark{1}Department of Electronic Engineering, Tsinghua University, Beijing, 100084, P. R. China\\
        E-mail: wtw12@mails.tsinghua.edu.cn, wangquan.thu@aliyun.com, stieizc.33@gmail.com, 370756595@qq.com}
	}
\maketitle
\section{Introduction}
Pedestrian detection is a key problem in computer vision,
with several applications that have the potential to positively
impact quality of life.\\\indent
Pedestrian detection is a difficult task from a machine
vision perspective. The lack of explicit models leads to the
use of machine learning techniques, where an implicit
representation is learned from examples.
In this proposal, we summarize the different perspectivs of studying
Pedestrian Detection problem with the help of several well written survey \cite{S1,S2,S3}.
And illustrate what we could do during the work.
\section{Previous Study}
There are numerous papers studying the problems from different perspectives, which includes:\\\indent
\textbf{ROI Selection}:
The simplest technique to obtain initial object location
hypotheses is the sliding window technique,
where detector windows at various scales and locations are shifted over the
image.\\\indent
\textbf{Classification}:
After a set of initial object hypotheses has been acquired,
further verification (classification) involves pedestrian
appearance models, using various spatial and temporal
cues\\\indent
\textbf{Tracking}:
There has been extensive work on the tracking of
pedestrians to infer trajectory-level information. One line
of research has formulated tracking as frame-by-frame
association of detections based on geometry and dynamics
without particular pedestrian appearance models 
\section{Current Resources}
We currently have over 50 papers on the Pedestrian Detection Problem,
all of them from the ICCV, ECCV, CVPR after 2011.
We have access to over 6 large data bases and more than 5 open source source codes on the problem.
\subsection{Related Work}
\subsubsection{Haar Wavelet-Based Cascade}
The Haar wavelet-based cascade framework provides
an efficient extension to the sliding window approach by
introducing a degenerate decision tree of increasingly
complex detector layers. 
\subsubsection{Neural Network Using Local Receptive Fields}
Adaptive local receptive fields (LRF) have been shown
to be powerful features in the domain of pedestrian
detection, in combination with a multilayer feed-forward
neural network architecture (NN/LRF).
\subsubsection{Histograms of Oriented Gradients with Linear SVM (HOG/linSVM)}
We follow the approach of Dalal and Triggs to model
local shape and appearance using well-normalized dense
histograms of gradient orientation
\subsubsection{Combined Shape-Texture-Based Pedestrian Detection}
We consider a monocular version of the real-time PROTECTOR system by cascading shape-based
pedestrian detection with texture-based pedestrian classification.
\subsection{Data Set}
Multiple public pedestrian datasets have been collected over the years, which includes:
INRIA, ETH, TUD-Brussels, Daimler (Daimler stereo), Caltech-USA, and KITTI.
\section{What Could We Do}
\subsection{Groups to Follow}
To efficiently get started and not to be drowned by the numerous papers,
we focus on several groups that are consistently researching the problems, which includes:
Dr. Xiaogang Wang \cite{W1,W2,W3,W4,W5,W6,W7} and 
Dr. Rodrigo Benenson from Max-Planck-Institut f\"ur Informatik \cite{R1,R2,R3,R4,R5,R6,R7}
\subsection{Time Table}
We sort out and finish the literature review by Week 9. 
We read and revise by week 11, after which we will focus on the developing of our won algorithms.
The research point could be applying deep learning, 
but no conclusion could be made before a full understanding of literature is accomplished.
\bibliographystyle{IEEEtran}
\bibliography{Ref}
\end{document}
